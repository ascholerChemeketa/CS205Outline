\documentclass[12pt,letterpaper]{article}
%% keep packages in alphabet order
\usepackage{amsmath}
\usepackage{amsfonts}
\usepackage{amssymb}
\usepackage{amsfonts}
\usepackage{amsmath}
\usepackage{amssymb}
\usepackage{array}
\usepackage[bottom]{footmisc}
\usepackage{caption}
\usepackage{color, colortbl}
%%\usepackage{csquotes} %causes conflict on windows miktex
\usepackage{enumerate}
\usepackage{enumitem}
\usepackage{fancyhdr}
\usepackage{fancyvrb}
\usepackage{float}
\usepackage{graphicx}
\PassOptionsToPackage{hyphens}{url}\usepackage[linktoc=all]{hyperref}
\usepackage[latin1]{inputenc}
\usepackage{listings}
\usepackage{longtable}
\usepackage{multicol}
\usepackage{multirow}
\usepackage{pdfpages}
\usepackage{pgf}
\usepackage{scrextend}
\usepackage{subfig}
\usepackage{titlesec}
\usepackage{titling}
\usepackage{verbatim}
\usepackage{ulem}
%% includes
% \include{emoticons}

%\newcommand{\LF}{}  % turn on to display large format
\ifdefined \LF
\usepackage[left=0.75in, top=0.75in, right=0.75in, bottom=0.75in, landscape]{geometry}  % for large format landscape
\else
\usepackage[left=0.75in, right=0.75in, top=0.75in, bottom=1.25in]{geometry}
\fi
%% New commands
\newcommand{\Ctrl}[1]{Control\,--\,#1}
%% change list bullets
\renewcommand{\labelitemi}{$\bullet$}
\renewcommand{\labelitemii}{$\rhd$}
\renewcommand{\labelitemiii}{--}
\renewcommand{\labelitemiv}{$\ast$}
\newcommand{\xdownarrow}[1]{%
	{\left\downarrow\vbox to #1{}\right.\kern-\nulldelimiterspace}
}
%% defaults for lstlistings
\lstset{
	basicstyle=\ttfamily,
	mathescape,
	xleftmargin=.25in,
}
%% setup default color scheme for references 
\hypersetup{
	colorlinks,
	citecolor=black,
	filecolor=black,
	linkcolor=black,
	urlcolor=red
}
\definecolor{bubbles}{rgb}{0.91, 1.0, 1.0}  %% exemplar, see http://latexcolor.com/
\definecolor{lightgray}{rgb}{0.9,0.9,0.9}  %%

\setlength{\headheight}{52pt}

\newbox\verbbox
\newlength\myverbindent 
\setlength\myverbindent{0.25in} % change this to change indentation. use \noindent to not indent
\makeatletter
\def\verbatim@processline{%
	\hspace{\myverbindent}\the\verbatim@line\par}
\makeatother
%%
\setlength\extrarowheight{2pt}  %% table padding, for readability
\interfootnotelinepenalty=10000  %% avoid footnotes being split across multiple pages
\fancyhf{}
\pagestyle{fancyplain}
\graphicspath{ {./images/} }

\newcolumntype{C}[1]{>{\centering\arraybackslash}m{#1}}  %% Use 'C' as column specifier to center text in tables

\setlength\arrayrulewidth{1pt} %% thicker lines for tables  %% put includes here. They tend to get reused a lot

\usepackage{lmodern}
\renewcommand*\familydefault{\sfdefault} % use lmodern for paragraphs

\chead{ \fancyplain{}{ CS 205 Course Description } }
\cfoot{ \fancyplain{}{\thepage} }
\setlength{\parskip}{1em}
\usepackage{indentfirst}

\renewcommand\thesection{} %no section numbers
\renewcommand\thesubsection{\arabic{subsection}} %subsection numbering does not include section

\begin{document}

    \section*{Overview}

    Useful samples for CS205

    \setcounter{tocdepth}{1}
    \tableofcontents
    \clearpage

    \section{Example Showing Impact of Cache Stride}\label{stride}

    \begin{verbatim}
    // example showing how stride through an array impacts execution time
    
    #include <stdio.h>
    #include <stdlib.h>
    #include <sys/time.h>
    
    #define MAX 1000
    int a[MAX][MAX][MAX];
    
    void
    test1() {
        for (int i=0; i<MAX; i++) {
            for (int j=0; j<MAX; j++) {
                for (int k=0; k<MAX; k++) {
                    a[i][j][k] = i+j+k;
                }
            }
        }
    }
    
    void
    test2() {
        for (int k=0; k<MAX; k++) {
            for (int j=0; j<MAX; j++) {
                for (int i=0; i<MAX; i++) {
                    a[i][j][k] = i+j+k;
                }
            }
        }
    }
    
    void
    getTime(struct timeval *tv)
    {
        if (gettimeofday(tv, 0) != 0) {
            printf("gettimeofday() failed. Aborting...\n");
            exit(EXIT_FAILURE);
        }
    }
    
    int
    main() 
    {
        struct timeval tv1, tv2, tv3;
        
        for (int i=1; i<3; i++) {
            getTime(&tv1);
            i%2 ? test1() : test2();
            getTime(&tv2);
            timersub(&tv2, &tv1, &tv3);
            printf("%s took %ld seconds and %ld microseconds\n",
                i%2 ? "test1()" : "test2()", tv3.tv_sec, tv3.tv_usec);
        }
        
        exit(EXIT_SUCCESS);
    }
    \end{verbatim}


    \noindent Output:
    \begin{verbatim}
    $ ./a.out
    test1() took 5 seconds and 660435 microseconds
    test2() took 122 seconds and 309618 microseconds
    \end{verbatim}

    \section{Example Showing Union and Bit Fields}
    Note: This is from PSU's CS 333 project 5, which isn't used anymore, but looked to add mode bits to the file system for implementing rudimentary protection.
    \begin{verbatim}
    union mode_t {
        struct {
            uint o_x : 1;
            uint o_w : 1;
            uint o_r : 1;  // other
            uint g_x : 1;
            uint g_w : 1;
            uint g_r : 1;  // group
            uint u_x : 1;
            uint u_w : 1;
            uint u_r : 1;  // user
            uint setuid : 1;
            uint     : 22; // pad
        } flags;
        uint asInt;
    };
        
    struct dinode {
        short type;           // File type
        short major;          // Major device number (T_DEV only)
        short minor;          // Minor device number (T_DEV only)
        short nlink;          // Number of links to inode in file system
        union mode_t mode;    // protection/mode bits
        uint size;            // Size of file (bytes)
        uint addrs[NDIRECT+1];   // Data block addresses
    };
    \end{verbatim}

    \section{Example of {\tt fork()}}
    \begin{verbatim}
    #include <stdio.h>
    #include <stdlib.h>
    #include <unistd.h>
    #include <string.h>
    #include <sys/types.h>
    #include <sys/wait.h>
    #include <errno.h>
        
    extern int errno;
        
    void
    print_error(char *msg)
    {
        fprintf(stderr, "%s: %s\n", msg, strerror(errno));
        exit(0);
    }
        
    pid_t
    Fork(void)
    {
        pid_t pid;
    
        if ((pid = fork()) < 0)
            print_error("fork system call failed");
        return pid;
    }
        
    int
    main(int argc, char *argv[])
    {
        int status;
        pid_t pid;
            
        pid = Fork();  // wrapper makes code easier to read
            
        if (pid == 0) {  // child process
            printf("Child process has UID %d and parent is %d\n", getpid(), getppid());
            sleep(10);
        }
        else {  // parent process
            printf("Parent process has UID %d and child is %d\n", getpid(), pid);
            pid = wait(&status);
            printf("child %d exited with status %d\n", pid, status);
        }
            
        exit(EXIT_SUCCESS);
    }
    \end{verbatim}

    \section{Example of {\tt exec()}}
    \begin{verbatim}
    #include <stdio.h>
    #include <unistd.h>
    #include <stdlib.h>
    #include <string.h>
    #include <errno.h>
        
    extern int errno;
        
    int
    main(int argc, char *argv[])
    {
        if (argc < 2) {
            fprintf(stderr, "You must provide a command to run!\n");
            fprintf(stderr, "Exiting...\n");
                exit(EXIT_FAILURE);
        }
            
        ++argv;  // a very cool trick
            
        // using execvp -- see Linux man page: https://linux.die.net/man/3/execvp
        execvp(argv[0], argv);
        // will only return on exec failure. i.e., bad command
        fprintf(stderr, "%s: %s\n", argv[0], strerror(errno));
            
        exit(EXIT_FAILURE);
    }
    \end{verbatim}
    \noindent Output:
    \begin{verbatim}
    $ ./a.out ls /
    bin   dev   home   lib64       media  pkgs  run   srv       sys  usr
    boot  disk  lib    libx32      mnt    proc  sbin  stash     tmp  var
    cat   etc   lib32  lost+found  opt    root  snap  swapfile  u    www
    $ ./a.out bad-command
    bad-command: No such file or directory
    $ ./a.out ls bad-dir
    ls: cannot access 'bad-dir': No such file or directory
    \end{verbatim}
    


    \section{Simple hello-world with syscalls for x86-64}
    \begin{verbatim}
        int
        main()
        {
            write(1, "hello world\n", 13);
            _exit(0);
        }
    \end{verbatim}
    
    The assembly code will look like this
    \begin{verbatim}
        .section .data
        string:
          .ascii "hello, world\n"
        string_end:
          .equ len, string_end - string
        .section .text
        .globl main
        main:
          ; First, call write(1, "hello, world\n", 13)
          movq $1, %rax      ; write is system call 1
          movq $1, %rdi      ; Arg1: stdout has descriptor 1
          movq $string, %rsi ; Arg2: hello world string
          movq $len, %rdx    ; Arg3: string length
          syscall            ;Make the system call
          ; Next, call _exit(0)
          movq $60, %rax     ; _exit is system call 60
          movq $0, %rdi      ; Arg1: exit status is 0
          syscall            ; Make the system call
    \end{verbatim}


    \section{Example Showing Embedded Assembly }
    
    Demonstrates using Intel's cpuid instruction.

    \begin{verbatim}
// More info @ https://www.felixcloutier.com/x86/cpuid
#include <stdio.h>
#include <stdlib.h>

void printChars(int c)
{
    // on Intel, integers are little endian, so chars are in reverse order.
    printf("%c", c & 0xff);
    printf("%c", (c >> 8) & 0xff);
    printf("%c", (c >> 16) & 0xff);
    printf("%c", (c >> 24) & 0xff);
    return;
}

void doCPUID(int choice)
{
    int a[8] = {0};

    switch (choice) {
    case 0:   // Vendor information
        __asm__("mov $0x0 , %eax\n\t");
        break;
    case 1:   // 2-4 are brand information
        __asm__("mov $0x80000002 , %eax\n\t");
        break;
    case 2:
        __asm__("mov $0x80000003 , %eax\n\t");
        break;
    case 3:
        __asm__("mov $0x80000004 , %eax\n\t");
        break;
    default:
        printf("Error! Aborting . . . \n");
        exit(EXIT_FAILURE);
    }

    // call cpuid instruction and store the results in a[].
    __asm__("cpuid\n\t");
    __asm__("mov %%eax, %0\n\t":"=r" (a[0]));
    __asm__("mov %%ebx, %0\n\t":"=r" (a[1]));
    __asm__("mov %%ecx, %0\n\t":"=r" (a[2]));
    __asm__("mov %%edx, %0\n\t":"=r" (a[3]));

    if (choice == 0) {  // vendor
//    printf("%d: ", a[0]);  // not interesting for us
    printChars(a[1]);
    printChars(a[3]);
    printChars(a[2]);
    printf("\n");
    }
    else {  // brand
    printf("%s", (char *)&a[0]);
    }
}

int main()
{
    for (int i=0; i<4; i++)
    doCPUID(i);
    printf("\n");
    exit(EXIT_SUCCESS);
}
    \end{verbatim}

\end{document}
